\documentclass[
  oneside,
  11pt, a4paper,
  footinclude=true,
  headinclude=true,
  cleardoublepage=empty
]{scrbook}

\usepackage{dissertation}

% ACRONYMS -----------------------------------------------------

%import the necessary package with some options
\usepackage[acronym,nonumberlist,nomain]{glossaries}

%enable the following to avoid links from the acronym usage to the list
%\glsdisablehyper

%displays the first use of an acronym in italic
\defglsdisplayfirst[\acronymtype]{\emph{#1#4}}

%the style of the Glossary
\glossarystyle{listgroup}

% set the name for the acronym entries page
\renewcommand{\glossaryname}{Acronyms}

%this shall be the last thing in the acronym configuration!!
\makeglossaries


% here are the acronym entries
\newacronym{mei}{MEI}{Mestrado em Engenharia Informática}
\newacronym{di}{DI}{Departamento de Informática}
\newacronym{um}{UM}{Universidade do Minho}

% ...

% these could go in an acronyms.tex file, and loaded with:
% \loadglsentries[\acronymtype]{Parts/Definitions/acronyms}
% when using this, you may want to remove 'nomain' from the package options

%% **MORE INFO** %%

%to add the acronyms list add the following where you want to print it:
%\printglossary[type=\acronymtype]
%\clearpage
%\thispagestyle{empty}

%to use an acronym:
%\gls{qps}

% compile the thesis in command line with the following command sequence:
% pdlatex dissertation.tex
% makeglossaries dissertation
% bibtex dissertation
% pdlatex dissertation.tex
% pdlatex dissertation.tex

% ----------------------------------------------------------------

% Title
\titleA{Intelligent Medical Image Analysis}
\titleB{} % (if any)
\subtitleA{A \textit{Deep Learning} approach to the }
\subtitleB{diagnosis of breast cancer} % (if any)

% Author
\author{João Pedro Pereira Fontes}

% Supervisor(s)
\supervisor{Professor Miguel Angel Guevara Lopez}
\cosupervisor{Professor Luís Gonzaga Mendes Magalhães}

% Date
\date{\myear} % change to text if date is not today

% Glossaries & Acronyms
%\makeglossaries  %  either use this ...
%\makeindex	   % ... or this

% Define Acronyms
%\input{sec/acronyms}
%\glsaddall[types={\acronymtype}]

\ummetadata % add metadata to the document (author, publisher, ...)

\begin{document}
  % Cover page ---------------------------------------
  \umfrontcover
  \umtitlepage

  % Add acknowledgements ----------------------------
  \chapter*{Acknowledgements}
    Write acknowledgements here

  % Add abstracts (en,pt) ---------------------------
  \chapter*{Abstract}
    Once medical images were scanned and uploaded to a computer, researchers began to create automated medical imaging systems. From the 1970s to the 1990s, medical imaging was performed with sequential application of low-level pixel processing and mathematical modeling to solve specific tasks such as organ segmentation. At the end of the 1990s, supervised techniques began to appear, where data extracted from the images were used to train models and classification systems. One example is the use of automated classifiers to build support systems for cancer detection and diagnosis. This pattern recognition and / or machine learning approach is still very popular and represented a shift from systems that were completely human-engineered to computer-trained systems with the use of specific (manually drawn) features and automatically extracted from the training data (example). The next step is that the algorithms directly learn the characteristics of the pixels of the images. This is the basic concept of Deep Learning algorithms: multi-layered models that transform input data (images) into outputs (e.g. the presence or absence of pathological lesions or cancer).

    This dissertation intends to study ways of using Deep Learning algorithms in the analysis of medical images, in particular for the classification of pathological lesions representative of cancer phenotypes.

  \cleardoublepage
  \chapter*{Resumo}
    Assim que foi possível digitalizar e carregar imagens médicas num computador, os investigadores começaram a criar sistemas automatizados para análise de imagens médicas. No intervalo dos anos 70 até aos anos 90, a análise de imagens médicas foi feita com aplicação sequencial de processamento de pixeis de baixo nível e modelação matemática para resolver tarefas específicas como, por exemplo, a segmentação de órgãos. No final dos anos 90, começam a aparecer as técnicas supervisionadas, onde os dados extraídos das imagens são usados para treinar modelos e sistemas de classificação. Um exemplo é o uso de classificadores automáticos para construir sistemas de apoio à deteção e diagnóstico do cancro. Esta abordagem de reconhecimento de padrões e/ou machine learning ainda é muito popular e representou uma mudança nos sistemas que eram completamente projetados por seres humanos para sistemas treinados por computadores com recurso ao uso de características especificas (manualmente desenhadas) e extraídas automaticamente dos dados de treino (exemplo). O passo seguinte a alcançar é que os algoritmos aprendam directamente as características dos pixeis das imagens. É este o conceito base dos algoritmos de Deep Learning: modelos (redes) compostos por muitas camadas que transformam dados de entrada (imagens) em saídas (por exemplo, a presença ou ausência de lesões patológicas ou cancro).

    Pretende-se com esta dissertação estudar formas de usar algoritmos de Deep Learning na análise de imagens médicas, em particular para a classificação de lesões patológicas representativas de fenotipos de cancro.

  % Summary Lists ------------------------------------
  \tableofcontents
  \listoffigures
  \listoftables
  \printglossary[type=\acronymtype]
  \clearpage
  \thispagestyle{empty}

  \pagenumbering{arabic}

  % CHAPTER - Introduction -------------------------
  \chapter{Introduction}
    This dissertation describing the  Master's work developed in the context of
    \gls{mei} held at \gls{di}, \gls{um}.\\
    Context,\\ motivation,\\ main aims	(objectives) \\ research hypothesis, (optional) \\ paper organization!

    %...

  % CHAPTER - State of the Art ---------------------
  \chapter{State of the art}
    State of the art review; related work

    \section{Basics/Background/Related work}
      %...

    \section{Summary}
      % ...

      \subsection{Conceptual map (Optional)}
        You may wish to use the \conexp{Concept-Explorer} tool.

  % CHAPTER - Problem and Challenges ---------------
  \chapter{The problem and its challenges}
    The problem and its challenges.

    \section{Proposed Approach - solution}
      % ...

      \subsection{System Architecture}
        % ...

  % CHAPTER - Contribution -------------------------
  \chapter{Development}
    % ...

    \section{Decisions}
      % ...

    \section{Implementation}
      % ...

    \section{Outcomes}
      Main result(s) and their scientific evidence

    \section{Summary}
      % ...

  % CHAPTER - Application -------------------------
  \chapter{Case Studies / Experiments}
    Application of main result (examples and case studies)

    \section{Experiment setup}
      % ...

    \section{Results}
      % ...

    \section{Discussion}
      % ...

    \section{Summary}
      % ...

  % CHAPTER - Conclusion/Future Work --------------
  \chapter{Conclusion}
    Conclusions and future work.

    \section{Conclusions}
      % ...

    \section{Prospect for future work}
      % ...

  \bookmarksetup{startatroot} % Ends last part.

  \addtocontents{toc}{\bigskip} % Making the table of contents look good.
  %\cleardoublepage

  %- Bibliography (needs bibtex) -%
  \bibliography{dissertation}

  % Index of terms (needs  makeindex) -------------
  \printindex

	% APPENDIX --------------------------------------
  \umappendix{Appendix}

    % Add appendix chapters
    \chapter{Support material}
      Auxiliary results which are not main-stream; or

    \chapter{Details of results}
      Details of results whose length would compromise readability of main text; or

    \chapter{Listings}
      Specifications and Code Listings: should this be the case; or

    \chapter{Tooling}
      Tooling: Should this be the case.

    %Anyone using \Latex\ should consider having a look at \TUG,
    %the \tug{\TeX\ Users Group}

  % Back Cover -------------------------------------------
  \umbackcover{
  NB: place here information about funding, FCT project, etc in which the work is framed. Leave empty otherwise.
  }

\end{document}
